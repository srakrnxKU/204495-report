เรียนด้วยความสัตย์จริง, ข้าพเจ้าไม่ใช่คนชอบเขียนเอกสาร และแน่นอนว่าข้าพเจ้าทุ่มเทเวลากับการเขียนรหัสคำสั่งมากกว่างานเอกสาร (จนถึงระดับที่รู้สึกว่าการเขียน $\LaTeX$ \footnote{ข้าพเจ้าขัดใจเป็นอย่างยิ่งกับการที่ตัว A และ E ถูกแสดงผลด้วยฟอนต์ TH Sarabun New เนื่องด้วยชุดคำสั่ง \texttt{textrm} ต่างจากสามตัวที่เหลือที่เป็็น Latin Modern} เพื่อรายงานช่างสนุกเสียเหลือเกิน!) แต่มีส่วนหนึ่งของเอกสารที่ข้าพเจ้าตั้งใจเขียนไม่แพ้กับชุดคำสั่ง--นั่นก็คือกิตติกรรมประกาศฉบับนี้

ข้าพเจ้าแปลกใจเล็กน้อยที่เห็นแม่แบบรูปเล่มรายงานโครงงานมีกิตติกรรมประกาศพิมพ์มาให้พร้อม--ข้าพเจ้าเข้าใจในความหวังดีของการใส่ตัวอย่างกิตติกรรมประกาศมาให้ ข้าพเจ้าเป็นคนแปลก (น่าจะใช้คำว่าแปลกกว่าชาวบ้านได้เต็มปาก) ที่ไม่อยากเขียนกิตติกรรมประกาศเพียงแค่พอเป็นพิธี ข้าพเจ้ารู้สึกว่าในเล่มโครงงานที่ข้าพเจ้าพยายามเขียนให้ตอบโจทย์ความคาดหวัง มีเพียงพื้นที่ตรงนี้ที่ข้าพเจ้าสามารถเขียนเรื่อยเปื่อยได้โดยไม่ต้องใส่ใจความคาดหวังอะไร

ประการแรก ข้าพเจ้าขอบคุณครอบครัวของข้าพเจ้า--มิใช่เพียงแค่ด้วย ``มารยาท'' ว่าชื่อแรกที่ต้องขอบคุณคือครอบครัว แม้ว่าจะมีบางกรณีที่เห็นต่างและสับสน แต่สิ่งหนึ่งที่วิเศษคือครอบครัวข้าพเจ้าเข้าใจในทางที่ข้าพเจ้าเลือกเดินเป็นอย่างยิ่ง บทสนทนาเรื่องเรียนต่อกับพ่อ สายโทรศัพท์ที่ถามว่ายังกินยาต้านซึมเศร้าอยู่หรือไม่จากแม่ และมุกตลกที่น้องชายข้าพเจ้าชอบเล่น คงตรึงใจอยู่ในความทรงจำสมัยวัยรุ่น

ข้าพเจ้าขอบคุณผศ.ดร.จิตร์ทัศน์ ฝักเจริญผล มิใช่ในฐานะที่ปรึกษาเพียงอย่างเดียว แต่ในฐานะมิตรสหายบนทวิตเตอร์ด้วย ไม่ว่าจะเป็นข้อความ บทสนทนา เวลากาแฟ หรือใดๆ ก็ตาม ข้าพเจ้าถูกผลักให้พบเจอมุมมองที่แตกต่าง ถูกท้าทายความคิด และมุมมองในปัจจุบันของข้าพเจ้า--ซึ่งเป็นตัวนิยามว่าข้าพเจ้า ``เป็นใคร''--ก็ได้รับการหล่อหลอมจากอาจารย์มามิใช่น้อย

ขอขอบพระคุณผศ.ดร.ธนาวินท์ รักธรรมานนท์ สำหรับการเป็นที่ปรึกษาร่วมในโครงงานนี้, ขอบพระคุณสำหรับการให้หนทางหลายๆ อย่าง ข้าพเจ้าอาจไม่อยู่ ณ จุดนี้หากไม่ได้รับโอกาสเหล่านั้น

ครั้นฝึกงาน ข้าพเจ้าเขียนย่อหน้าที่ขอบคุณเพื่อนด้วยการนำชื่อมาเรียงต่อกันตามลำดับของสิ่งที่ขอบคุณ (กล่าวคือหากข้าพเจ้าขอบคุณเพื่อนสำหรับ \(t_1, t_2, \dots, t_n\) คำขอบคุณที่ \(t_i\) นั้นจะเจาะจงไปยังเพื่อนคนที่ \(ni+1, n \in \mathbb{I}^+\) เป็นพิเศษ) คงได้เวลาที่ข้าพเจ้าจะขอบคุณพวกเขาอีกครั้ง อย่างละเอียด ให้สมกับความนับถือที่ข้าพเจ้ามีให้กับเขาเหล่านั้น
\begin{itemize}
    \item รวิสรา--เราคงไม่ต้องพูดอะไรมาก เราอยากเขียนย่อหน้านี้ให้มันยาวเป็นร้อยบรรทัด ให้ครบกับทุกอย่างที่ให้เรา, เราอยากเขียนเพียง :) ในย่อหน้านี้ด้วยเชื่อว่าไม่มีทางที่เราจะเขียนข้อความเป็นร้อยบรรทัดให้แกได้จบ, แต่ขอบคุณ ขอบคุณจริงๆ คงไม่พอหรอก แต่ขอบคุณ
    \item กิตติยา--กิต ฮิว ชิบิ ชิไบ หิว หิ๋ว หรือชื่อไหนก็ตาม, ขอบคุณสำหรับหลายๆ อย่างเช่นเดียวกัน คำว่า ``คนที่วิเศษ'' จะยังอยู่กับเราในวันที่ไม่ใช่วันที่ดีที่สุดของเรา, ขอบคุณสำหรับคำนี้และอีกหลายอย่าง มัน ``วิเศษ'' และมาจากคนที่ก็ ``วิเศษ'' เหมือนกัน
    \item รล--มุข เรียวเล็น รหล๋ (อ่านว่ารอ-หลอ) รูปหล่อ เรือหลวง แรกเหลียญ หรือขื่อใดก็ตาม, ปลายปากกาของเพื่อนที่สร้างสรรค์งานวาดช่วยให้เรามีแรงบันดาลใจอย่างไม่น่าเชื่อ (ขออนุญาตพูดว่าขอขอบคุณด้วยเสียงรวิสรา) และในฐานะที่เราอาจจะไม่ใช่คนที่พูดเก่งด้วยกันทั้งคู่ ทุกการ ``บรีบ'' นั้นเรารับรู้ได้ว่ามาจากใจจริงๆ และอยากจะบอกว่ามันช่วยให้โลกนี้อยู่ง่ายขึ้นมานิดนึงเสมอ ขอขอบคุณ (พูดด้วยเสียงรวิสราอีกครั้ง)
    \item แพรว--ณิชา, แผว, แผ๋ว, แพร๋ว (สามชื่อหลัง ถ้าไม่ผิดหลักการสะกด ก็ไม่ใช่การออกเสียงที่ถูก), มันมักจะมีมุมมุมหนึ่งที่เรารู้สึกห่างกัน และมุมเดียวกันนั้นเองที่เรารู้สึกว่าเราจะไม่มีทางห่างกันได้เลย, ขอบคุณสำหรับที่ที่ทำให้ไม่เคยไม่เป็นตัวเองแม้แต่วินาทีเดียว, ไปจัดทริปเฮดเหรอบที่ล้านด้วยกันนะ
    \item ข้าพเจ้าอยากขอหนึ่งจุดหัวข้อตรงนี้ในการตั้งคำถามว่าเพราะเหตุใดเพื่อนในภาควิชาถึงมีชื่อเล่นมากมายเหลือเกิน
    \item อ้น--มีชุดความคิดชุดหนึ่งที่เราดีใจที่เห็นในตัวอ้นไม่ใช่น้อย เป็นชุดความคิดที่ทำให้เราชื่นชมในตัวอ้นอยู่, ขอให้เติบโตอย่างแข็งแกร่ง และเป็นกำลังใจให้เสมอ
    \item วรชัยและมอร์แกน--แม้จะห่างกันขึ้นจากเวลาและระยะทาง รวมถึงภาระส่วนตัว แต่ดีใจเสมอที่ครั้งหนึ่งได้มาเจอกันแบบงงๆ และอยู่ด้วยกันสี่ปีจนจบแบบงงๆ เช่นกัน (คำว่างงๆ ที่ปรากฎในครั้งหลัง ขยายคำว่าจบ หรือคำว่าอยู่ ขอให้ตีความเอง)
    \item นิว--คุณเป็นหลายอย่างเหมือนกันในช่วงเวลาที่ผ่านมานี้ คุณเป็นธิงค์ แทงค์ (think tank), คุณเป็นคนวิพากษ์ คุณเป็นอาหารกระตุ้นสมอง และเป็นคนที่ทำให้ผมเห็นมุมองใหม่มิใช่น้อย ที่สำคัญ ขอบคุณสำหรับส่วนหนึ่งของการทำให้ผมยอมรับในตัวเองได้
    \item พรมนัส--ขอบคุณสำหรับทุกมีมแดงค์ (dank memes) และการคุยกันทางปรัชญาที่ชวนให้รู้สึกเฮฮาอยู่ไม่ใช่น้อย, คุณเป็นคนที่สนุกมากคนนึงเลยนะ
    \item เปรม และจุ้ย--ขอบคุณที่มาเป็นส่วนหนึ่งของกันเมื่อไม่นานมานี้ และดีใจที่ได้มีคุณเป็นหนึ่งในความทรงจำช่วงมหาวิทยาลัย
    \item เนยสด--แม้เราจะรู้จักกันผ่านทวิตเตอร์มานาน แต่ช่วงเวลาที่เราเพิ่งได้ร่วมงานกันและสิงสถิตในแล็บด้วยกันเป็นช่วงเวลาที่มีความสุข ขอบคุณสำหรับทุกความห่วงใยและเมสเสจที่ทักเข้ามาเป็นระยะ
\end{itemize}
แน่นอนว่ารายชื่อนี้ไม่ได้กล่าวถึงทุกคนในภาควิชา--ขอถือโอกาสนี้กล่าวขอบคุณทุกคนจากใจจริงอีกครั้งหนึ่ง และขอถือโอกาสนี้ในการขอบคุณมิตรทุกท่านระดับชั้นมัธยมศึกษาที่ยังคงไม่ปล่อยให้โคจรหายออกจากกันไป (และขออภัยสำหรับการแบกแล็ปท็อปไปทำงานทุกครั้งที่นัดกินข้าว--นี่ไง งานเสร็จแล้วนะ)

ข้าพเจ้าขอขอบคุณสมาชิกกลุ่มวิจัยเชิงทฤษฎี, บรรยากาศการทำงานที่นี่วิเศษมาก ข้าพเจ้าไม่คิดว่าจะหาบรรยากาศการทำงานแบบนี้ได้จากที่ไหน กลุ่มวิจัยเป็นทั้งที่ทำงาน ที่พักผ่อน ที่ที่อบอุ่นสำหรับข้าพเจ้า, คงเป็นการยากที่ข้าพเจ้าจะบรรยายถึงความทรงจำในห้อง 805, บนโซฟาสีน้ำเงินตัวเก่า, บนเบาะถุงถั่ว (bean bag) สีแดง, และตรงที่นั่งทำงาน ขอขยายคำขอบคุณนี้ไปถึงสมาชิกรับเชิญของกลุ่มวิจัยฯ ทุกท่านที่เข้ามาเยี่ยมเยือนห้องนี้เป็นระยะ

ขอบคุณทีมการเรียนรู้เชิงประสงค์ร้ายทุกคน--
วัชรพัฐ เมตตานันท, พงศกร อัจฉริยศักดิ์ชัย, มณฑล จรัสตระกูล--การร่วมงานกับคนเก่งนับเป็นเกียรติอย่างยิ่งสำหรับข้าพเจ้า, การได้ยืนท่ามกลางผู้คนเหล่านั้นช่วยผลักดันให้ข้าพเจ้าเรียนรรู้โลกในแง่ที่ตรงกับความจริงมากขึ้น และสร้างความเชื่อมั่นว่าข้าพเจ้าจะไม่มีใครมาหยุดได้

ขอบคุณในความเชื่อที่จะเปลี่ยนโลกด้วยงานวิจัย ของอาจารย์จากสำนักวิชาวิทยาศาสตร์และเทคโนโลยีสารสนเทศ สถาบันวิทยสิริเมธี: อาจารย์สรณะ นุชอนงค์, อาจารย์ธีรวิทย์ วิไลประสิทธิ์พร, อาจารย์ปรเมษฐ์ มนูญพงศ์, อาจารย์ศุภศรณ์ สุวจนกรณ์ รวมถึงอาจารย์โชคชัย เลี้ยงสุขสันต์ คณบดีคนแรกของสำนักวิชา และขอขอบคุณบุคลากร และนิสิตของสำนักวิชาฯ ที่พร้อมผลักดันความเชื่อนี้ในสภาพแวดล้อมการทำงานที่ดี

ในการเดินทางอันยาวไกล ความเหนื่อยล้าเป็นเรื่องธรรมดา ขอขอบคุณภัทรวี ศรีสันติสุข สำหรับความอบอุ่นหัวใจแบบที่ยากแก่การอธิบาย เสียงของเอิ๊ตปลอบประโลมในวันที่ใจอ่อนล้า, สะกดให้อยู่ในห้วงอารมณ์ในวันที่เร่งรีบ, รอยยิ้มของเอิ๊ตทำให้ข้าพเจ้าอยากยิ้มได้งดงามและเป็นความสุขให้คนอื่นได้แบบที่เอิ๊ตเป็น

ขอบคุณตุ๊กตาทุกตัวของข้าพเจ้าที่ให้กอดในวันที่ใจอ่อนล้าโดยไม่ลังเล \footnote{หากตุ๊กตาของข้าพเจ้าลุกมาบอกข้าพเจ้าว่าลังเล ข้าพเจ้าคงวิ่งหนี} ขอฝากความขอบคุณนี้ไปยังตุ๊กตาที่หอพักและที่กลุ่มวิจัยฯ เป็นพิเศษ: บิบิ จี้ คุณหอยทาก วลาดิเมียร์ บิบินอน ปิ๊บปู่ว์ ปิ๊บปู่ว์จิ๋ว และคาปู

ขอบคุณทุกท่านที่ร่วมอุดมการณ์และชุดความคิดเดียวกับข้าพเจ้า สำหรับคำบอกย้ำที่ทรงพลังว่าข้าพเจ้าไม่ได้อยู่เพียงคนเดียว, ในโลกอันกว้างใหญ่ ข้าพเจ้ามิอาจเลี่ยงความรู้สึกว่าตัวเองโดดเดี่ยว หากแต่คนเหล่านี้คือเครื่องย้ำเตือนข้าพเจ้าว่าโลกใบนี้มิได้เปล่าเปลี่ยวและโหดร้ายอย่างที่คิด หลายท่านเป็น\textit{ผู้มาก่อนกาล} ผู้บุกเบิกแนวคิด ผู้เสนอสิ่งที่ต้องห้าม ผู้เชื่อในสิ่งที่เป็นไปไม่ได้ ผู้ต่อสู้ ผู้เรียกร้อง ผู้ยืนหยัด ดั่งวลีว่า ``คนยังคงยืนเด่นโดยท้าทาย''

ขอขอบคุณหลายสิ่งในชีวิตที่ช่วยให้ข้าพเจ้าอยู่บนโลกนี้ได้ง่ายขึ้น และหลับตานอนได้สบายกว่าที่เคยเป็น: ตู้อบผ้าที่ตั้งใกล้หอข้าพเจ้า, เตารีดไอน้ำเสี่ยวหมี, ร้านขายยาต้านซึมเศร้าที่ราคาไม่แพง, พนักงานบริการส่งอาหารทุกท่าน กลไกเล็กๆ เหล่านี้ช่วยให้ข้าพเจ้าเหนื่อยน้อยลง

สุดท้ายนี้ข้าพเจ้าอยากขอบคุณเด็กผู้ชายคนหนึ่ง: แทน ในระดับชั้นประถม  เราจำได้ว่าแทนในตอนนั้นอยากเป็นนักวิทยาศาสตร์ หวังว่านักวิจัยคอมพิวเตอร์คงใกล้เคียงกันอยู่นะ ดีใจด้วย! แทนในระดับชั้นมัธยม เก่งมากนะที่ยืนหยัดตัวเองให้มาในทางที่ชอบแม้ว่าจะมีหันเหเล็กน้อย ทำได้แล้วนะ

รายงานฝึกงานชิ้นนี้คงเป็นเหมือนสูจิบัตรของกระบวนสุดท้ายในการแสดงที่ชื่อว่าปริญญาตรีวิศวกรรมคอมพิวเตอร์ ในไม่ช้าบาตองในมือข้าพเจ้าจะส่งสัญญาณตัดจบ (cutoff) ครั้งสุดท้ายในการแสดงนี้ ข้าพเจ้าจะหันหลังกลับไปพบเสียงปรบมือ พบผู้คนที่รายล้อมและชื่นชมกับความสำเร็จของข้าพเจ้า หากมีเสียงปรบมือใดก็ตาม ข้าพเจ้าจะขอผายมือไปยังข้างหลังข้าพเจ้า ให้เสียงปรบมือเหล่านั้นดังถึงทุกคนในกิตติกรรมประกาศฉบับนี้ และข้าพเจ้าขอน้อมรับทุกเสียงตอบรับที่มีให้กับข้าพเจ้าจากใจจริง ขอขอบคุณ

\vskip 20pt

\hfill\begin{minipage}
    {\dimexpr 5cm}
    \begin{center}
        ศิระกร ลำใย\\
        ผู้จัดทำ
    \end{center}
    \xdef\tpd{\the\prevdepth}
\end{minipage}