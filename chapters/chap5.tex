\section{สรุป}

\subsection{สรุปผลการทำโครงงาน}

ผลการศึกษาเบื้องต้นพบว่าผลการเสริมความแข็งแกร่งแบบจำลองนั้นเป็นที่น่าพอใจในการป้องกันการโจมตีด้วยขั้นตอนวิธี PGD อย่างไรก็ตามเป็นที่น่าประหลาดใจว่าแบบจำลองยังคงไม่ทนทานต่อการโจมตีด้วยขั้นตอนวิธี FGSM เทียบเท่าขั้นตอนวิธีเสริมความแข็งแกร่งอื่น

ทั้งนี้ ขั้นตอนวิธีที่เสนอตอบโจทย์การลดเวลาฝึกสอนแบบจำลองเพื่อเพิ่มความแข็งแกร่งเป็นอย่างดี

\subsection{ผลลัพธ์ที่ได้}

ได้ขั้นตอนวิธีสำหรับฝึกสอนแบบจำลองเพื่อเพิ่มความแข็งแกร่งแบบร่นเวลา

\subsection{ประโยชน์ที่ได้รับจากการปฏิบัติงาน}

เป็นการศึกษาขั้นต้นของการป้องกันแบบจำลองจากการเรียนรู้เชิงประสงค์ร้าย ซึ่งจะนำไปต่อยอดเป็นงานวิจัยต่อไป

\section{ปัญหาอุปสรรคและแนวทางการแก้ไขปัญหา}

\subsection{ปัญหาที่พบระหว่างการทำโครงงาน}

\begin{itemize}
    \item ข้อจำกัดทางด้านทรัพยากรการคำนวน
    \item ข้อจำกัดทางด้านการทดลองซ้ำได้ และการปรับค่าการทดลองซ้ำ
    \item ข้อจำกัดทางด้านความแปรปรวนทางอารมณ์
\end{itemize}

\subsection{แนวทางการแก้ไขปัญหา}

\begin{itemize}
    \item ใช้เงินแก้ปัญหา กล่าวคือเช่าระบบเครือข่ายกลุ่มเมฆสำหรับการฝึกสอนแบบจำลอง
    \item ปรับกระบวนรหัส (refactor) ให้สามารถทดลองหลายแบบได้อย่างยืดหยุ่น
    \item พบจิตแพทย์ และทานยา
\end{itemize}
