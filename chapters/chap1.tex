\section{ความเป็นมาและความสําคัญของปัญหา}

แบบจำลองจักรกลเรียนรู้ (Machine Learning models) นั้นถูกใช้อย่างกว้างขวางในปัจจุบัน อย่างไรก็ตาม แบบจำลองใดๆ นั้นอาจมีความผิดพลาดต่อการทำการโจมตีประสงค์ร้าย (Adversarial attacks) เพื่อจงใจให้ผลลัพธ์ที่แบบจำลองนั้นคาดเดามีความผิดพลาดจากผลลัพธ์ที่ควรจะเป็น

ในการเรียนรู้เชิงตัวแปรเสริม (parameter-based learning) นั้น ตัวแปรเสริม (parameters) ค่าน้ำหนัก (weights) บนแบบจำลองการเรียนรู้เชิงลึก (deep Learning models) เป็นตัวกำหนดความฉลาดของแบบจำลอง อาจมีตัวแปรเสริมบางชุดที่ทำให้แบบจำลองมีช่องโหว่ต่อการโจมตีประสงค์ร้าย การโจมตีนั้นอาจเกิดจากการเพิ่มสัญญาณรบกวนซึ่งผ่านการคำนวน (calculated artefacts) เข้าสู่ข้อมูลรับเข้า (inputs) ซึ่งทำให้ความผิดพลาดของแบบจำลองในการพยากรณ์คำตอบนั้นเปลี่ยนไปอย่างชัดเจน 

โครงงานวิศวกรรมคอมพิวเตอร์นี้มุ่งหวังจะนำตัวแปรเสริมบนแบบจำลองมาสร้างภาพแสดง (visualise) ถึงจุดโหว่ในการพยากรณ์ใดๆ ของแบบจำลอง เพื่อลดความเสียหายอันอาจเกิดขึ้นได้จากการโจมตีแบบจำลองขณะถูกใช้งานจริง

\section{วัตถุประสงค์ของการศึกษา}
\noindent
โครงงานนี้มีวัตถุประสงค์และเป้าหมายดังนี้

\begin{enumerate}
    \item สร้างแบบจำลองเชิงลึก (Deep Learning models) ซึ่งสามารถถูกโจมตีประสงค์ร้าย (Adversarial attacks) ได้
    \item นำแบบจำลองในข้อ (1) มาสร้างเป็นรูปภาพแสดง (visualisation) เพื่อหาจุดโหว่ต่อการโจมตี รวมถึงคาดเดาแนวโน้มการโจมตีที่เป็นไปได้
    \item ใช้ความรู้ในข้อ (2) สร้างแบบจำลองที่ทนทาน (prone) ต่อการโจมตีมากขึ้น
\end{enumerate}

\section{ขอบเขตของการทําโครงงาน}
\noindent
โครงงานนี้มีขอบเขตการดำเนินงานดังนี้

\begin{enumerate}
    \item สร้างแบบจำลองเชิงลึก (Deep Learning models) ซึ่งสามารถถูกโจมตีประสงค์ร้าย (Adversarial attacks) ได้
    \item นำแบบจำลองในข้อ (1) มาสร้างเป็นรูปภาพแสดง (visualisation) เพื่อหาจุดโหว่ต่อการโจมตี รวมถึงคาดเดาแนวโน้มการโจมตีที่เป็นไปได้
    \item ใช้ความรู้ในข้อ (2) สร้างแบบจำลองที่ทนทาน (prone) ต่อการโจมตีมากขึ้น
\end{enumerate}

\section{ระยะเวลาและแผนดําเนินงาน}
การดำเนินงานของโครงการจะใช้วิธีจัดการงาน (workflow) แบบเอไจล์ (agile) เพื่อมุ่งเน้นประสิทธิภาพในการทำงานและสร้างระบบการทำงานที่เหมาะสมต่อการดำเนินการในระยะเวลาและปัจจัยการดำเนินงานที่ไม่อาจคาดเดาได้
การทำงานจะใช้วิธีการแบ่งรอบการวนทวน (iteration) โดยมุ่งเน้นให้แต่ละรอบการวนทวนมีความก้าวหน้าของงานในปริมาณที่เหมาะสมกับเวลาและข้อจำกัดต่างๆ
หนึ่งรอบการวนทวนนั้นกินระยะเวลาสองสัปดาห์ดังแสดงในตารางที่ \ref{iteration-timetable} และจะประกอบด้วยขั้นตอนต่อไปนี้
\begin{enumerate}
    \item ประชุมสรุป (iteration meeting) หนึ่งถึงสองครั้งต่อสัปดาห์
    \item เขียนรอบปูมย้อนหลัง (backlog) และหยิบมาทำตามจำนวนที่เหมาะสม
    \item กิจกรรมมองย้อนรอบการวนทวนด้วยตนเอง (self-retrospective) เพื่อพิจารณาความเหมาะสมในการทำงาน และปรับปรุงการทำงานในรอบการวนทวนต่อไป
\end{enumerate}
อย่างไรก็ดี เพื่อเป็นการตั้งเป้าหมายงานในระยะยาว โครงงานนี้จะมีวิสัยทัศน์ผลิตภัณฑ์ (product vision) โดยคร่าวตามขอบเขตการดำเนินงาน และในทุกประมาณ 4-6 รอบการวนทวน จะมีการวางแผนปล่อยผลิตภัณฑ์ (release planning) หนึ่งครั้งเพื่อสรุปงานออกมาเป็นความคืบหน้าที่จับต้องได้อันเกิดจากการทำงานในกลุ่มรอบการวนทวนนั้น

\begin{table}[]
    \centering
    \begin{tabular}{r|cccc|ccc}
    \hline \hline
     & \multicolumn{4}{c|}{2562} & \multicolumn{3}{c}{2563} \\
     & ก.ย. & ต.ค. & พ.ย. & ธ.ค. & ม.ค. & ก.พ. & มี.ค. \\ \hline
    รอบการวนทวนที่ 1 & / &  &  &  &  &  &  \\
    รอบการวนทวนที่ 2 & / &  &  &  &  &  &  \\
    รอบการวนทวนที่ 3 &  & / &  &  &  &  &  \\
    รอบการวนทวนที่ 4 &  & / &  &  &  &  &  \\
    รอบการวนทวนที่ 5 &  &  & / &  &  &  &  \\
    รอบการวนทวนที่ 6 &  &  & / &  &  &  &  \\
    แผนการปล่อยงานที่ 1 &  &  & / &  &  &  &  \\
    รอบการวนทวนที่ 7 &  &  &  & / &  &  &  \\
    รอบการวนทวนที่ 8 &  &  &  & / &  &  &  \\
    รอบการวนทวนที่ 9 &  &  &  &  & / &  &  \\
    รอบการวนทวนที่ 10 &  &  &  &  & / &  &  \\
    แผนการปล่อยงานที่ 2 &  &  &  &  & / &  &  \\
    รอบการวนทวนที่ 11 &  &  &  &  &  & / &  \\
    รอบการวนทวนที่ 12 &  &  &  &  &  & / &  \\
    รอบการวนทวนที่ 13 &  &  &  &  &  &  & / \\
    รอบการวนทวนที่ 14 &  &  &  &  &  &  & / \\
    แผนการปล่อยงานที่ 3 &  &  &  &  &  &  & / \\
    \hline \hline
    \end{tabular}
    \caption{ตารางแสดงรอบการวนทวนและช่วงเวลา}
    \label{iteration-timetable}
\end{table}

\section{ประโยชน์ที่คาดว่าจะได้รับ}
\begin{enumerate}
    \item เข้าใจถึงพื้นฐาน หลักการทำงาน และระบบจักรกลเรียนรู้แบบต่างๆ
    \item เข้าใจถึงจุดอ่อนของระบบจักรกลเรียนรู้ในแต่ละกรณี
    \item สามารถโจมตีระบบจักรกลเรียนรู้ เพื่อสร้างระบบจักรกลเรียนรู้ที่ทนทานต่อการโจมตีได้
\end{enumerate}

\section{คํานิยามศัพท์เฉพาะ}
\begin{itemize}
    \item \textbf{จักรกลเรียนรู้} (machine learning) คือระบบ หรือโค้ด หรือโปรแกรมคอมพิวเตอร์ที่เรียนรู้โครงสร้างของชุดคำถามและคำตอบโดยมิจำเป็นต้องทำการโปรแกรมลำดับการทำงานอย่างชัดแจ้ง (explicitly) 
    \item \textbf{การเรียนรู้เชิงโจมตี} (adversarial learning) หมายถึงการศึกษาถึงการโจมตีแบบจำลอง (model) ของจักรกลเรียนรู้ (machine learning)
\end{itemize}