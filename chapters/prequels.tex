\newgeometry{top=2in}
\begin{center}
    {\LARGE โครงงานวิศวกรรมคอมพิวเตอร์} \par
    \vspace{1em}
    {\Large ขั้นตอนวิธีคลัสเตอร์เพื่อเสริมความทนทานต่อการโจมตีแบบจำลองการเรียนรู้เชิงลึก} \par
    {\Large Cluster Method to Strengthen Adversarial Defence on Deep Learning Models} \par
    \par
    \vskip 15em
    {\Large
        นาย ศิระกร ลำใย รหัสนิสิต 5910500023 \par
    }
    \vfill
    {\Large
        โครงงานวิศวกรรมนี้ เป็นส่วนหนึ่งของการศึกษาตามหลักสูตรวิศวกรรมศาสตรบัณฑิต\\
        ภาควิชาวิศวกรรมคอมพิวเตอร์\\
        คณะวิศวกรรมศาสตร์ มหาวิทยาลัยเกษตรศาสตร์\\
        ปีการศึกษา 2562\\
        ลิขสิทธิ์เป็นของคณะวิศวกรรมศาสตร์ มหาวิทยาลัยเกษตรศาสตร์
    }\par
\end{center}

\newpage

\begin{center}
    {\LARGE โครงงานวิศวกรรมคอมพิวเตอร์} \par
    \vspace{1em}
    {\Large ขั้นตอนวิธีคลัสเตอร์เพื่อเสริมความทนทานต่อการโจมตีแบบจำลองการเรียนรู้เชิงลึก} \par
    {\Large Cluster Method to Strengthen Adversarial Defence on Deep Learning Models} \par
    \par
    \vskip 15em
    {\Large
        นาย ศิระกร ลำใย รหัสนิสิต 5910500023 \par
    }
    \vfill
    ได้รับการพิจารณาเห็นชอบจากภาควิชาวิศวกรรมคอมพิวเตอร์ให้นับเป็นส่วนหนึ่งของการศึกษา\\
    ตามหลักสูตร วิศวกรรมศาสตรบัณฑิต สาขาวิชาวิศวกรรมคอมพิวเตอร์\\
    มหาวิทยาลัยเกษตรศาสตร์
    \par
    \end{center}

    \noindent
    อาจารย์ที่ปรึกษา \hfill
    ......................................................................................วันที่ ..... เดือน ................... พ.ศ. ............\\
    \hspace*{2cm}(ผศ. ดร. จิตร์ทัศน์ ฝักเจริญผล)\\
    อาจารย์ที่ปรึกษาร่วม \hfill
    ......................................................................................วันที่ ..... เดือน ................... พ.ศ. ............\\
    \hspace*{2cm}(ผศ. ดร. ธนาวินท์ รักธรรมานนท์)\\
    หัวหน้าภาควิชา \hfill
    ......................................................................................วันที่ ..... เดือน ................... พ.ศ. ............\\
    \hspace*{2cm}(รศ. ดร. พันธุ์ปิติ เปี่ยมสง่า)
\newpage
\pagenumbering{roman}

\noindent
ศิระกร ลำใย 2563: ขั้นตอนวิธีคลัสเตอร์เพื่อเสริมความทนทานต่อการโจมตีแบบจําลองการเรียนรู้เชิงลึก,
ปริญญาวิศวกรรมศาสตร์บัณฑิต (สาขาวิศวกรรมคอมพิวเตอร์) ภาควิชาวิศวกรรมคอมพิวเตอร์ คณะวิศวกรรมศาสตร์ มหาวิทยาลัยเกษตรศาสตร์\\
อาจารย์ที่ปรึกษาโครงงาน: ผศ.ดร.จิตร์ทัศน์ ฝักเจริญผล

\begingroup
\let\clearpage\relax
\chapter*{บทคัดย่อ}
\addcontentsline{toc}{chapter}{บทคัดย่อภาษาไทย}
\endgroup

\newpage

\noindent
Sirakorn Lamyai 2020: Cluster Method to Strengthen Adversarial Defence on Deep Learning Model,
Bachelor of Engineering (Computer Engineering), Department of Computer Engineering, Faculty of Engineering, Kasetsart University\\
Project advisor: Assoc. Prof. Dr. Jittat Fakcharoenphol

\begingroup
\let\clearpage\relax
\chapter*{Abstract}
\addcontentsline{toc}{chapter}{Abstract}
\endgroup
