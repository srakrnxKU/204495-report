\documentclass{cpereport}

\begin{document}
\chapter{บทนำ}

\section{ความเป็นมาและความสําคัญของปัญหา}

แบบจำลองจักรกลเรียนรู้ (Machine Learning models) นั้นถูกใช้อย่างกว้างขวางในปัจจุบัน อย่างไรก็ตาม แบบจำลองใดๆ นั้นอาจมีความผิดพลาดต่อการทำการโจมตีประสงค์ร้าย (Adversarial attacks) เพื่อจงใจให้ผลลัพธ์ที่แบบจำลองนั้นคาดเดามีความผิดพลาดจากผลลัพธ์ที่ควรจะเป็น

ในการเรียนรู้เชิงตัวแปรเสริม (parameter-based learning) นั้น ตัวแปรเสริม (parameters) ค่าน้ำหนัก (weights) บนแบบจำลองการเรียนรู้เชิงลึก (deep Learning models) เป็นตัวกำหนดความฉลาดของแบบจำลอง อาจมีตัวแปรเสริมบางชุดที่ทำให้แบบจำลองมีช่องโหว่ต่อการโจมตีประสงค์ร้าย การโจมตีนั้นอาจเกิดจากการเพิ่มสัญญาณรบกวนซึ่งผ่านการคำนวน (calculated artefacts) เข้าสู่ข้อมูลรับเข้า (inputs) ซึ่งทำให้ความผิดพลาดของแบบจำลองในการพยากรณ์คำตอบนั้นเปลี่ยนไปอย่างชัดเจน 

โครงงานวิศวกรรมคอมพิวเตอร์นี้มุ่งหวังจะนำตัวแปรเสริมบนแบบจำลองมาสร้างภาพแสดง (visualise) ถึงจุดโหว่ในการพยากรณ์ใดๆ ของแบบจำลอง เพื่อลดความเสียหายอันอาจเกิดขึ้นได้จากการโจมตีแบบจำลองขณะถูกใช้งานจริง

\section{วัตถุประสงค์ของการศึกษา}
\noindent
โครงงานนี้มีวัตถุประสงค์และเป้าหมายดังนี้

\begin{enumerate}
    \item สร้างแบบจำลองเชิงลึก (Deep Learning models) ซึ่งสามารถถูกโจมตีประสงค์ร้าย (Adversarial attacks) ได้
    \item นำแบบจำลองในข้อ (1) มาสร้างเป็นรูปภาพแสดง (visualisation) เพื่อหาจุดโหว่ต่อการโจมตี รวมถึงคาดเดาแนวโน้มการโจมตีที่เป็นไปได้
    \item ใช้ความรู้ในข้อ (2) สร้างแบบจำลองที่ทนทาน (prone) ต่อการโจมตีมากขึ้น
\end{enumerate}

\section{ขอบเขตของการทําโครงงาน}
\noindent
โครงงานนี้มีขอบเขตการดำเนินงานดังนี้

\begin{enumerate}
    \item สร้างแบบจำลองเชิงลึก (Deep Learning models) ซึ่งสามารถถูกโจมตีประสงค์ร้าย (Adversarial attacks) ได้
    \item นำแบบจำลองในข้อ (1) มาสร้างเป็นรูปภาพแสดง (visualisation) เพื่อหาจุดโหว่ต่อการโจมตี รวมถึงคาดเดาแนวโน้มการโจมตีที่เป็นไปได้
    \item ใช้ความรู้ในข้อ (2) สร้างแบบจำลองที่ทนทาน (prone) ต่อการโจมตีมากขึ้น
\end{enumerate}

\section{ระยะเวลาและแผนดําเนินงาน}
ในช่วงแรกของการทำโครงงาน แผนการดำเนินงานนั้นจะใช้ในรูปแบบของรบวนทวนซ้ำ (iteration) ตามกรรมวิธีการดำเนินงานแบบเอไจล์ (agile) ซึ่งประกอบไปด้วยขั้นตอนการวนทวนดังนี้\dots

\section{ประโยชน์ที่คาดว่าจะได้รับ}
\begin{enumerate}
    \item เข้าใจถึงพื้นฐาน หลักการทำงาน และระบบจักรกลเรียนรู้แบบต่างๆ
    \item เข้าใจถึงจุดอ่อนของระบบจักรกลเรียนรู้ในแต่ละกรณี
    \item สามารถโจมตีระบบจักรกลเรียนรู้ เพื่อสร้างระบบจักรกลเรียนรู้ที่ทนทานต่อการโจมตีได้
\end{enumerate}

\section{คํานิยามศัพท์เฉพาะ}
\begin{itemize}
    \item \textbf{ระบบจักรกลเรียนรู้} (machine learning) คือระบบ หรือโค้ด หรือโปรแกรมคอมพิวเตอร์ที่เรียนรู้โครงสร้างของชุดคำถามและคำตอบโดยมิจำเป็นต้องทำการโปรแกรมลำดับการทำงานอย่างชัดแจ้ง (explicitly) 
    \item \textbf{การเรียนรู้เชิงโจมตี} (adversarial learning) หมายถึงศาสตร์
\end{itemize}

\chapter{ทฤษฎีและงานวิจัยที่เกี่ยวข้อง}
\section{จักรกลเรียนรู้}
ระบบจักรกลเรียนรู้ (machine learning) อาจนิยามได้ว่าเป็นระบบที่ไม่ต้องมีการป้อนข้อมูล หรือวิธีทำงาน เข้าไปยังโค้ดโปรแกรมอย่างชัดแจ้ง (explicitly) โดยระบบดังกล่าวจะถูกฝึกสอนด้วยชุดของข้อมูลหรือประสบการณ์ (experience) และปรับตัวเองให้ส่งออกคำตอบซึ่งอิงจากประสบการณ์ที่ตนเองเคยได้เรียนรู้มา

หากจะกล่าวให้ละเอียด เราสามารถนิยามโปรแกรมซึ่งสามารถทำการ\textit{เรียน}ได้ดังนี้ \cite{Mitchell97}

\begin{definition}
โปรแกรมใดๆ เรียน (learn) จากประสบการณ์ (experience) $E$ บนงาน (task) $T$ และการวัดประสิทธิผล (performance measurement) $P$ หากประสิทธิผลบน $T$ ซึ่งถูกวัดโดย $P$ เพิ่มขึ้นตามประสบการณ์ $E$
\end{definition}

\bibliography{references/chap2} 
\bibliographystyle{ieeetr}
\end{document}